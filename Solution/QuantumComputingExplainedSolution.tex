\documentclass{article}
\usepackage[utf8]{inputenc}
\usepackage[T1]{fontenc}
\usepackage{imakeidx}
\usepackage{amsmath}
\usepackage{amssymb}
\usepackage{graphicx}
\usepackage{physics}

\graphicspath{ {images/} }

\makeindex
\title{Quantum Computing Explained Solutions}
\author{Ricardo Alcaraz Fraga}
\date{February 2021}

\begin{document}

\begin{titlepage}
\maketitle
\end{titlepage}

\tableofcontents
\clearpage

\section{Introduction}
This document has the intention to show the solutions to the exercises presented in Quantum Computing Explained by David McMahon, the exercies are explained and in some cases R or Python code is presented.

\section{A brief introduction to information theory}
\textbf{1.1} How many bits are necessary to represent the alphabet using a binary code if we only allow uppercase characters? How about if we allow both uppercase and lowercase character?\\ \\
\textbf{Solution}\\
Let's consider that \textit{n} is the number of characters in an alphabet, we would need \textit{x} bits so that \\ 

\begin{center}
$2^x \geq n$\\
$x \geq \log_2 n$\\
\end{center}
The english alphabet has 26 characters, so $n=26$, therefore\\
\begin{center}
$\log_2 26 \approx 4.7$
\end{center}
And since we can't  have 4.7 bits we have to take the next integer, that is 5.\\
\begin{center}
$\therefore{}$ We need 5 bits
\end{center}
How about if we allow both uppercase and lowercase characters? This implies that $n=26 * 2 = 52$ characters. We could solve same equations as above, or we can just remember that\\
\begin{center}
$2^{x + 1} = 2 * 2^x$
\end{center}
Knowing this we can see that\\
\begin{center}
$2^6 \geq 52$\\
$64 \geq 52$\\
$\therefore{}$ We need 6 bits
\end{center}
\textbf{1.2} Describe how you can create an OR gate using NOT gates and AND gates.\\ \\
\textbf{Solution}\\
To create a 1 bit OR gate we need 2 NOT gates and a NAND gate (which is an AND gate followed by a NOT gate), this gates have to be connected as follows (the not gates are going to be created with a NAND gate which receives the signal two times)\\ \\
\begin{center}
\includegraphics[scale = 0.75]{or using nands.png}
\end{center}
We can see that the result of the circuit is $\overline{\overline{x} \ \overline{y}}$. To develop this we have to remember one Boole's Algebra rule
\begin{center}
$\overline{xy} = \overline{x} + \overline{y}$
\end{center}
Using this rule we can see that \\
\begin{align*}
\overline{\overline{x} \ \overline{y}} = \overline{\overline{x}} + \overline{\overline{y}}\\
\overline{\overline{x}} + \overline{\overline{y}} = x + y
\end{align*}
\textbf{1.3} A kilobyte is 1024 bytes. How many messages can it store?\\ \\
\textbf{Solution}\\
This is a very straight forward solution, all that we have to remember is\\
\begin{align*}
1 byte &= 8 bits\\
1024 bytes &= 1024 * 8\\
&= 8192 bits
\end{align*}
\begin{center}
$\therefore{}$ 1 kilobyte can store $2^{8192}$ messages\\
\end{center}
\textbf{1.4} What is the entropy associated with the tossing of a fair coin?\\ \\
\textbf{Solution}\\
Let's remember that a fair coin implies the following\\
\begin{center}
$P(\{heads\}) = P(\{tails\}) = 0.5$
\end{center}
So, being $X$ the event of tossing a fair coin\\
\begin{align*}
H(X) &= -\sum p_i \log_2 p_i\\
&= -(0.5 \log_2 0.5 + 0.5 \log_2 0.5)\\
&= -(-0.5 - 0.5)\\
&= 1
\end{align*}
\textbf{1.5} Suppose that $X$ consists of the characters A, B, C, D that occur in a signal with respective probabilities 0.1, 0.4, 0.25, 0.25, what is  the Shanon's entropy?\\ \\
\textbf{Solution}\\
Given the probabilities \\
\begin{center}
$P(\{A\}) = 0.1$\\
$P(\{B\}) = 0.4$\\
$P(\{C\}) = 0.25$\\
$P(\{D\}) = 0.25$\\
\end{center}
The Shanon's entropy is the following\\
\begin{align*}
H(X) &= - \sum p_i log_2 p_i\\
&= -(0.1 \log_2 0.1 + 0.4 \log_2 0.4 + 0.25 \log_2 0.25 + 0.25 \log_2 0.25)\\
&= -(0.332 - 0.528 - 0.5 - 0.5)\\
&= 1.86
\end{align*}
\textbf{1.6} A room full of people has incomes distributed in the following way\\ \\
$n(25.5) = 3$\\
$n(30) = 5$\\
$n(42) = 7$\\
$n(50) = 3$\\
$n(63) = 1$\\
$n(75) = 2$\\
$n(90) = 1$\\ \\
What is the most probable income? What is the average income? What is the variance of this distribution?\\ \\
\textbf{Solution}\\
Starting by the most probable income, we can calculate the probability of all incomes, but we don't  have to, knowing that the total people $n = 22$, we can see that the probability of an income is\\
\begin{center}
$\frac{n_j}{n}$
\end{center}
Where $n_j$ is the count of occurences of that income, so the higher the occurences the higher the probability, knowing that the higher occurence is 7, $n(42)$ is the most probable income with a value of $\frac{7}{22}$.\\
The average income is given by the expresion \\
\begin{align*}
\mu &= \sum j p_j\\
&= 25.5(0.13) + 30(0.22) + 42(0.31) + 50(0.13) + 63(0.04) + 75(0.09) + 90(0.04)\\
&= 44.25
\end{align*}
The variance of the distribution can be calculated in many ways, but I'm going to use the next formula\\
\begin{center}
$\sigma^2 = \frac{\sum X^2}{N} - \mu^2$
\end{center}
Where $X$ is the random variable, N is the number of observations, and $\mu$ is the mean of the distribution. Therefore, the variance  is\\
\begin{align*}
\sigma^2 &= \frac{1}{22}(1950.75 + 4500 + 12348 + 7500 + 3969 + 11250 + 8100) - 44.25^2\\
&= 2255.3522 - 1958.0625 = 297.289
\end{align*}

\section{Qubits and Quantum States}
\textbf{2.1} A quantum state is in the state\\ \\
\begin{center}
$\frac{(1 - i)}{\sqrt{3}}\ket{0} + \frac{1}{\sqrt{3}}\ket{1}$\\
\end{center}
If a measurement is made, what is the probability the system is in the state $\ket{0}$ or in state $\ket{1}$?\\ \\
\textbf{Solution}\\
The probability that the system is in state $\ket{0}$ is\\
\begin{center}
$\mid\alpha\mid ^2$, where $\alpha = \frac{(1 - i)}{\sqrt{3}}$\\
\end{center}
And so\\
\begin{align*}
\mid\alpha\mid ^2 &= (\alpha)(\alpha)^*\\
&= (\frac{1 - i}{\sqrt{3}})(\frac{1 + i}{\sqrt{3}})\\
&= \frac{1^2 - i^2}{\sqrt{3}}\\
&= \frac{2}{3}\\
\end{align*}
So the probability that the system is in state $\ket{0}$ is $\frac{2}{3}$, and as we know (if the system is normalized) the probability of the state being in state $\ket{1}$ is\\
\begin{center}
$1 - \mid\alpha\mid ^2$\\
\end{center}
But we don't know if the system is normalized, and so we are going to calculate the probability of the system being in state $\ket{1}$ as we did for state $\ket{0}$\\
Let $\beta = \frac{1}{\sqrt{3}}$\\
\begin{align*}
\mid\beta\mid^2 &= (\beta)(\beta)^*\\
&= (\frac{1}{\sqrt{3}})(\frac{1}{\sqrt{3}})\\
&= \frac{1}{3}
\end{align*}
\textbf{2.2} Two quantum states are given by\\
\begin{center}
$\ket{a} = \begin{pmatrix} -4i \\ 2 \end{pmatrix}$,   $\ket{b} = \begin{pmatrix} 1 \\ -1 + i \end{pmatrix}$
\end{center}
\textit{(A)} Find $\ket{a + b}$\\
\textit{(B)} Calculate $3\ket{a} - 2\ket{b}$\\
\textit{(C)} Normalize $\ket{a}$, $\ket{b}$\\ \\
\textbf{Solution}\\
\textit{(A)}
\begin{align*}
\ket{a + b} &= \begin{pmatrix} -4i \\ 2 \end{pmatrix} + \begin{pmatrix} 1 \\ -1 + i \end{pmatrix}\\
&= \begin{pmatrix} 1-4i \\ 2-1+i \end{pmatrix}\\
&= \begin{pmatrix} 1-4i \\ 1+i \end{pmatrix}\\
\end{align*}
\textit{(B)} 
\begin{align*}
3\ket{a} - 2\ket{b} &= 3\begin{pmatrix} -4i \\ 2 \end{pmatrix} - 2\begin{pmatrix} 1 \\ -1 + i \end{pmatrix}\\
&= \begin{pmatrix} -12i \\ 6 \end{pmatrix} - \begin{pmatrix} 2 \\ -2 + 2i \end{pmatrix}\\
&= \begin{pmatrix} 2-12i \\ 6-2+2i \end{pmatrix}\\
&= \begin{pmatrix} 2-12i \\ 4+2i \end{pmatrix}\\
&= 2\begin{pmatrix} 1-6i \\ 2+i \end{pmatrix}\\
\end{align*}
\textit{(C)}
Normalizing $\ket{a}$\\ \\
Let $\ket{\overline{w1}} = {\ket{a}}$\\
\begin{align*}
\bra{\overline{w1}}\ket{\overline{w1}} &= \begin{pmatrix} -4i \\ 2 \end{pmatrix}^\dag \begin{pmatrix} -4i \\ 2 \end{pmatrix}\\
&= \begin{pmatrix} -4i & 2 \end{pmatrix}^* \begin{pmatrix} -4i \\ 2 \end{pmatrix}\\
&= \begin{pmatrix} 4i & 2 \end{pmatrix} \begin{pmatrix} -4i \\ 2 \end{pmatrix}\\
&= -16i^2 + 4\\
&= 16 + 4\\
&= 20
\end{align*}
Then we have\\
\begin{align*}
\ket{w1} &= \frac{\ket{\overline{w1}}}{\sqrt{\bra{a}\ket{a}}}\\
&= \frac{1}{\sqrt{20}}\begin{pmatrix} -4i \\ 2\end{pmatrix}\\
&= \frac{1}{2\sqrt{5}}\begin{pmatrix} -4i \\ 2\end{pmatrix}\\
&= \frac{1}{\sqrt{5}}\begin{pmatrix} -2i \\ 1\end{pmatrix}\\
\end{align*}
Now we have $\ket{a}$ normalized, which is $\ket{w1}$ Now we normalize $\ket{b}$\\
Let\\
\begin{center}
$\ket{\overline{w_2}} = \ket{b} - \frac{\bra{\overline{w_1}}\ket{b}}{\bra{\overline{w_1}}\ket{\overline{w_1}}} \ket{\overline{w_1}}$
\end{center}
Let's start by calculating $\bra{\overline{w_1}}\ket{b}$\\
\begin{align*}
\bra{\overline{w_1}}\ket{b} &= \begin{pmatrix} 4i & 2\end{pmatrix} \begin{pmatrix} 1 \\ -1+i\end{pmatrix}\\
&= 4i -2 + 2i\\
&= -2 + 6i
\end{align*}
We already calculated $\bra{\overline{w1}}\ket{\overline{w1}}$ which is 20, so then we calculate $\ket{\overline{w_2}}$\\
\begin{align*}
\ket{\overline{w_2}} &= \ket{b} - \frac{\bra{\overline{w_1}}\ket{b}}{\bra{\overline{w_1}}\ket{\overline{w_1}}} \ket{\overline{w_1}}\\
&= \begin{pmatrix} 1 \\ -1+i \end{pmatrix} - \frac{-2+6i}{20} \begin{pmatrix} -4i \\ 2 \end{pmatrix}\\
&= \begin{pmatrix} 1 \\ -1+i \end{pmatrix} + \frac{2-6i}{20} \begin{pmatrix} -4i \\ 2 \end{pmatrix}\\
&= \begin{pmatrix} 1 \\ -1+i \end{pmatrix} + \frac{2(1-3i)}{20} \begin{pmatrix} -4i \\ 2 \end{pmatrix}\\
&= \begin{pmatrix} 1 \\ -1+i \end{pmatrix} + \frac{1-3i}{10} \begin{pmatrix} -4i \\ 2 \end{pmatrix}\\
&= \begin{pmatrix} 1 \\ -1+i \end{pmatrix} + \frac{1}{10} \begin{pmatrix} -4i+12i^2 \\ 2-6i \end{pmatrix}\\
&= \begin{pmatrix} 1 \\ -1+i \end{pmatrix} + \begin{pmatrix}\frac{-12-4i}{10} \\ \frac{2-6i}{10}\end{pmatrix}\\
&= \begin{pmatrix} 1 \\ -1+i \end{pmatrix} + \begin{pmatrix}\frac{-6-2i}{5} \\ \frac{1-3i}{5}\end{pmatrix}\\
&= \begin{pmatrix} 1 + \frac{-6-2i}{5} \\ -1+i + \frac{1-3i}{5} \end{pmatrix}\\
&= \frac{1}{5} \begin{pmatrix} -1-2i \\ -4+2i \end{pmatrix}
\end{align*}
Now that we have this, we can normalize this state, to do so we need to 


\printindex
\end{document}
